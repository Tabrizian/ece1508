% --------------------------------------------------------------
% This is all preamble stuff that you don't have to worry about.
% Head down to where it says "Start here"
% --------------------------------------------------------------
 
\documentclass[12pt]{article}
 
\usepackage[margin=1in]{geometry} \usepackage{amsmath,amsthm,amssymb}
\usepackage{minted}
 
\newcommand{\N}{\mathbb{N}}
\newcommand{\Z}{\mathbb{Z}}
 
\newenvironment{theorem}[2][Theorem]{\begin{trivlist}
\item[\hskip \labelsep {\bfseries #1}\hskip \labelsep {\bfseries #2.}]}{\end{trivlist}}
\newenvironment{lemma}[2][Lemma]{\begin{trivlist}
\item[\hskip \labelsep {\bfseries #1}\hskip \labelsep {\bfseries #2.}]}{\end{trivlist}}
\newenvironment{exercise}[2][Exercise]{\begin{trivlist}
\item[\hskip \labelsep {\bfseries #1}\hskip \labelsep {\bfseries #2.}]}{\end{trivlist}}
\newenvironment{problem}[2][Problem]{\begin{trivlist}
\item[\hskip \labelsep {\bfseries #1}\hskip \labelsep {\bfseries #2.}]}{\end{trivlist}}
\newenvironment{question}[2][Question]{\begin{trivlist}
\item[\hskip \labelsep {\bfseries #1}\hskip \labelsep {\bfseries #2.}]}{\end{trivlist}}
\newenvironment{corollary}[2][Corollary]{\begin{trivlist}
\item[\hskip \labelsep {\bfseries #1}\hskip \labelsep {\bfseries #2.}]}{\end{trivlist}}

\newenvironment{solution}{\begin{proof}[Solution]}{\end{proof}}
 
\begin{document}
 
% --------------------------------------------------------------
%                         Start here
% --------------------------------------------------------------
 
\title{Homework 5}
\author{Iman Tabrizian\\ %replace with your name
ECE1508}

\maketitle

In order to deploy the mentioned topology I used the general structure like below: 

\begin{enumerate}
	\item Checking for existence of a previous switches / hosts
	\item If hosts / switches doesn't exist boot the VM 
	\item Creating OVS bridge named \textit{br1} in each switch / host
	\item Going through each of the connections between vms and switches and checking weather connection existed before
	\item Creating new vxlan tunnel if the connection didn't exist
\end{enumerate}


Now I want to describe each of the functions and what they do step by step. \\

\begin{minted}{python}
def getVM(vmName):
    nova = getNovaClient()
    vmName = overlay_config.username + '-' + vmName
    server = nova.servers.find(name=vmName)
    return server
\end{minted}

This function receives \textit{vmName} as input and returns the vm object returned by nova client. \\
\begin{center}
	\line(1,0){250}
\end{center}

\begin{minted}{python}
def getNovaClient():
    nova = novaClient.Client(
        overlay_config.username,
        overlay_config.password,
        overlay_config.tenant_name,
        overlay_config.auth_url,
        region_name=overlay_config.region,
        no_cache=True
        )
    return nova
\end{minted}

This function simply returns the nova client by filling in the appropriate credentials from the \textit{overlay\_config} file. \\

\begin{center}
	\line(1,0){250}
\end{center}

\begin{minted}{python}
def bootVM(vmName):
    assert type(vmName) in (str, unicode), "'vmName' is type %s" % type(vmName)

    # Pre-pend vmName with your username
    vmName = overlay_config.username + '-' + vmName

    logger.info("Creating VM %s" % vmName)

    # STUDENTS FILL THIS PART OUT
    nova = getNovaClient()
    net_name = overlay_config.tenant_name + '-net'
    net = nova.networks.find(label=net_name)
    flavor = nova.flavors.find(name=overlay_config.flavor)
    image = nova.images.find(name=overlay_config.image)
    server = nova.servers.create(
        vmName,
        image,
        flavor,
        key_name=overlay_config.key_name,
        security_groups=[overlay_config.username],
        nics=[{'net-id': net.id}]
        )

    waitUntilVMActive(server)

    return server
\end{minted}

This function creates a new vm prefixed using the username. Then, it retrieves the flavor, network and image objects from using the nova client and finally it creates the server using \textit{nova.servers.create}. In the end, function waits until the vm becomes active and then returns.

\begin{center}
	\line(1,0){250}
\end{center}
\begin{minted}{python}
def setOverlayInterface(hostVMObj, hostOverlayIP):
    logger.info("Setting overlay for %s with IP %s" %
                        (hostVMObj.name, hostOverlayIP))
    networkName = overlay_config.tenant_name + '-net'

    hostVMIP = hostVMObj.networks.get(networkName)[0]
    hostSSH = getSSHSession(hostVMIP, 'ubuntu', 'savi')

    # Ensure OVS daemon is up and running
    waitUntilOVSActive(hostSSH)

    # STUDENTS FILL THIS PART OUT
    runCommandOverSSH(hostSSH, 'sudo ovs-vsctl --may-exist add-br br1')
    if hostOverlayIP is not None:
        runCommandOverSSH(hostSSH, 'sudo ovs-vsctl --may-exist add-port br1 br1-internal -- set interface br1-internal type=internal')
        runCommandOverSSH(hostSSH, 'sudo ifconfig br1-internal %s/24 mtu 1450 up' % hostOverlayIP)
\end{minted}

This function simply adds an ovs bridge. If the \textit{hostOverlayIP} parameter is not equal to \textit{None}, it will also add one port to this bridge and configure its mtu as mentioned in the assignment.

\begin{center}
	\line(1,0){250}
\end{center}
\begin{minted}{python}
def connectNodes(node1, node2):
    logger.info("Making VXLAN links between %s and %s" % (node1.name, node2.name))
    networkName = overlay_config.tenant_name + '-net'

    node1IP = node1.networks.get(networkName)[0]
    node1SSH = getSSHSession(node1IP, 'ubuntu', 'savi')

    node2IP = node2.networks.get(networkName)[0]
    node2SSH = getSSHSession(node2IP, 'ubuntu', 'savi')

    # Ensure OVS daemon is up and running in both nodes
    waitUntilOVSActive(node1SSH)
    waitUntilOVSActive(node2SSH)

    # STUDENTS FILL THIS PART OUT
    vni = generateVNI()
    runCommandOverSSH(node1SSH, 'sudo ovs-vsctl add-port br1 %s
            -- set interface %s type=vxlan options:remote_ip=%s options:key=%s'
            % (
                node1.name + '-' + node2.name,
                node1.name + '-' + node2.name,
                node2IP,
                vni
                ))
    runCommandOverSSH(node2SSH, 'sudo ovs-vsctl add-port br1 %s
            -- set interface %s type=vxlan options:remote_ip=%s options:key=%s'
            % (
                node2.name + '-' + node1.name,
                node2.name + '-' + node1.name,
                node1IP,
                vni))



\end{minted}

This function connects two hosts using VXLAN tunneling. First, it creates a \textit{VNI} using the helper function provided in order to ensure uniqueness of future VNIs. Then, adds a port to the previously created bridge (br1). The name of this port is in this format \textit{(source)- (dest)}

\begin{center}
	\line(1,0){250}
\end{center}
\begin{minted}{python}
def deployOverlay():
    print "In deployOverlay()"

    # Dictionaries to map switch/host names to their Nova VM objects
    createdSwitches = {}
    createdHosts = {}
    createdConnections = []

    # STUDENTS FILL THIS PART OUT
    for switch in overlay_config.topology.keys():
        switchVMObj = None

        if switch in createdSwitches:
            switchVMObj = createdSwitches[switch]
        else:
            switchVMObj = bootVM(switch)
            createdSwitches[switch] = switchVMObj
            setOverlayInterface(switchVMObj, None)

        for endpoint in overlay_config.topology[switch]:
            endpointVMObj = None
            if type(endpoint) is str:
                if endpoint + '-' + switch in createdConnections:
                    continue
                elif switch + '-' + endpoint in createdConnections:
                    continue
                else:
                    createdConnections.append(switch + '-' + endpoint)

                if endpoint in createdSwitches:
                    endpointVMObj = createdSwitches[endpoint]
                else:
                   endpointVMObj = bootVM(endpoint)
                   createdSwitches[endpoint] = endpointVMObj
                   setOverlayInterface(endpointVMObj, None)

            elif type(endpoint) is tuple:
                if endpoint[0] + '-' + switch in createdConnections:
                    continue
                elif switch + '-' + endpoint[0] in createdConnections:
                    continue
                else:
                    createdConnections.append(switch + '-' + endpoint[0])
                if endpoint in createdHosts:
                    endpointVMObj = createdHosts[endpoint[0]]
                else:
                   endpointVMObj = bootVM(endpoint[0])
                   createdHosts[endpoint] = endpointVMObj
                   setOverlayInterface(endpointVMObj, endpoint[1])

            connectNodes(switchVMObj, endpointVMObj)
\end{minted}

This function first checks whether the switch/host was previously created or not. If it wasn't previously created, it creates a new VM with appropriate overlay IP addresses. Then, checks wheather the connection exists or not (in order to avoid loops). If connection didn't existed before, it creates a new connection using \textit{connectNodes} function and stores the connection in order to avoid creating loops in the future.

\begin{center}
	\line(1,0){250}
\end{center}

\begin{minted}{python}
def listOverlay():
    print "In listOverlay()"

    # STUDENTS FILL THIS PART OUT
    networkName = overlay_config.tenant_name + '-net'

    nova = getNovaClient()
    for server in nova.servers.list():
        if server.name.startswith(overlay_config.username + '-'):
            vmIP = server.networks.get(networkName)[0]
            print server.name + ':' + ' ' + server.id + ' (' + vmIP + ')'
\end{minted}

This function simply iterates over all of the VMs available in this project and prints the names of the ones that begin with \textit{(username)-}


\begin{center}
	\line(1,0){250}
\end{center}
\begin{minted}{python}
def cleanupOverlay():
    print "In cleanupOverlay()"

    # STUDENTS FILL THIS PART OUT
    nova = getNovaClient()
    for server in nova.servers.list():
        if server.name.split('-')[0] == overlay_config.username:
            print server.name + ' deleted!'
            server.delete()
\end{minted}

This function simply iterates over the all of the VMs available in this project and deletes all of the VMs that match the current user.

\begin{center}
	\line(1,0){250}
\end{center}




 
 
\end{document}
